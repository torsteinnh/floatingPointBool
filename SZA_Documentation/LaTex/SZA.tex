\documentclass{article}

\usepackage{amsmath}
\usepackage{amsthm}
\usepackage{amssymb}

\title{SZA - Signed Zero Algebra \\
       \large A New Take On Propositional Logic}
\author{Thomas W Thorbjørnsen}
\date{07.04.2019}

\begin{document}
  \maketitle
  \tableofcontents

  \newpage
  \pagenumbering{gobble}

  \section{Defining the Algebraic Structure}
    \subsection*{Motivation}
      As one usually does on a Saturday evening, we found ourselves suddenly engrossed in a vivid discussion
      about the sociatal value of the number zero. Whilst discussing the differences between null, zero, nill and naught, and their nonevident
      linguistic differance in norwegian; a comment was made regarding the existence of both a positive and a negative zero in java.
      This proposition was by some found perposterus, and prompted an immediate investigation.
      Upon said investigation, its truthfulness was found evident, further bewildering the doubtful subjects regarding its arithmetical implementation.
      Addition, subtraction and multiplication was all applied, and the results organised and examined.
      What thenceforth emerged bore a striking resembalance to boolean algebra,
      and the expansion and elaboration of said results is the subject of this paper.

    \subsection*{Finding the Appropriate Field}

  \newpage

  \section{Building Logic Conjunctions}
    \subsection*{Primitive Conjunctions}

    \subsection*{Composite Conjunctions}

    \subsubsection*{How To Do Computation}
    
  \newpage

  \section{An Application}
    \subsection{Creating a Low Level Arithmetic Machine}
      \subsubsection*{Making a Full Bit Adder}

      \subsubsection*{Extending the Machine}

\end{document}
