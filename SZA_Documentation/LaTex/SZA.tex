%Making an article and importing unicode%
\documentclass[12pt]{report}
\usepackage[utf8x]{inputenc}

%Importing packages for Mathematics%
\usepackage{amsmath}
\usepackage{amsthm}
\usepackage{amssymb}

%Titles and stuff%
\title{\Huge SZA Signed Zero Algebra \\
       \Large A New Take On Propositional Logic}
\author{{Thomas W Thorbjørnsen} \\
        {Eirik Wittersø}\\
        {Benjamin Benjaminsen}\\
        {Torstein Nordgård-Hansen}}
\date{07.04.2019}

\pagenumbering{arabic}

%Package for fixing padding%
\usepackage{geometry}
\geometry{left=2.5cm,right=2.5cm,top=2.5cm,bottom=2.5cm}

\begin{document}

\maketitle
\large
\tableofcontents
\newpage

\chapter{Introduction}
  \section{Preface}
  \section{Motivation}
    As one usually does on a Saturday evening,
    we found ourselves suddenly engrossed in a vivid discussion
    about the societal value of the number zero.
    Whilst discussing the differences between null, zero, nil and naught,
    and their non-evident linguistic difference in Norwegian;
    a comment was made regarding the existence of both a positive and a negative zero in java.
    This proposition was by some found preposterous, and prompted an immediate investigation.
    Upon said investigation, its truthfulness was found evident, further bewildering the doubtful subjects
    regarding its arithmetical implementation.
    Addition, subtraction and multiplication was all applied, and the results organized and examined.
    What thenceforth emerged bore a striking resemblance to Boolean algebra,
    and the expansion and elaboration of said results is the subject of this paper.

\chapter{Defining the Algebraic Structure}
  \section{Defining the Object}
  \section{Making an Algebra}
  \subsection{The One and Only structure, $\mathbb{Z}_2$}
  \section{A Flashback to Boolean Algebra}
  \section{Linking SZA to Boolean Algebra}


\newpage

\chapter{Building Logic Conjunctions}
  \section{Primitive Conjunctions}

  \section{Composite Conjunctions}

  \section{How To Do Computations}
    \subsection{From SZA to "Normal" Mathematics}
    \subsection{Proofs of given Composite Conjuctions}

\newpage

\chapter{An Application}
  \section{Creating a Low Level Arithmetic Machine}
    \subsection{Making a Full Bit Adder}

    \subsection{Extending the Machine}

\end{document}
